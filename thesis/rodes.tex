\chapter{Extension to Random Dynamical Systems}
\label{ch-rodes}

As another important family of differential equations, \emph{random ordinary differential equations (RODEs)} are closely related to both ODEs and SDEs.
A system of RODEs is simply a set of ODEs with a stochastic process in its vector field functions \citep{kloeden2007pathwise}, while an SDE system can be analyzed using its RODE counterpart \citep{sussmann1978gap, imkeller2001conjugacy}.
Since RODEs are pathwise ODEs, the Laplace mean-field approximation described in \refchapter{\ref{ch-laplace-approximation}} can then be used to infer the states and parameters of the RODEs, or equivalently, the corresponding SDEs.

This chapter is organized as follows.
\refsection{\ref{sec-rodes}} gives a brief introduction to RODEs.
In \refsection{\ref{sec-rodes-laplace}}, the Laplace mean-field approximation technique is applied to RODEs to devise an \emph{ensemble}-like \citep[\refsection{16.6}]{murphy2012machine} solution to infer the states and parameters for diffusion processes.
We demonstrate the performance and accuracy of the solution empirically by comparing with other state-of-art techniques in \refchapter{\ref{ch-experiments}}.

\section{Random ordinary differential equations}
\label{sec-rodes}

Adopting the definition from \cite{kloeden2007pathwise}, RODEs are simply ODEs with a stochastic process in their vector fields.
Let $\mvector{f}: \R^K\times\R^W\mapsto\R^K$ be a continuous function, and $(\mvector{\zeta}_t)_{t\in [0,T]}$ be an $\R^W$ valued stochastic process with continuous sample paths defined on the complete probability space $\probspace$.  
For all $\mvector{\omega} \in \mvector{\Omega}$, a $K$-dimensional RODE defined as
\begin{align}
    \frac{d\dymx}{dt} = \mvector{f}(\dymx, \mvector{\zeta}_t(\mvector{\omega}))
    \label{eq-rodes}
\end{align}
is a \emph{non-autonomous} ODE system
\begin{align}
    \dymdx = \frac{d\dymx}{dt} = \mvector{F}_{\mvector{\omega}}(\mvector{x}, t) = \mvector{f}(\dymx, \mvector{\omega}_t)
    \label{eq-rodes-odes}
\end{align}
An example \citep{grune2001pathwise} of a scalar RODE with additive noise is given by
\begin{align}
    \frac{dx(t)}{dt} = -x + \cos{(W_t(\omega))}
    \label{eq-rodes-example}
\end{align}
where $W_t$ is a one-dimensional Wiener process.
A RODE example with multiplicative noise can be defined similarly but is not considered in this work.

Following \cite{kloeden2007pathwise}, to ensure the existence of a unique solution for the initial value problem defined in \refsection{\ref{sec-odes}} on the finite time interval $[0,T]$, we typically assume that $\mvector{f}$ is arbitrarily smooth, i.e.\ it is infinitely differentiable in its variables, and thus is locally \emph{Lipschitz} in $\mvector{x}$. 
Since the stochastic process is usually only \emph{H{\"o}lder continuous} in time, the vector fields of the non-autonomous ODEs $\mvector{F}_{\mvector{\omega}}(\mvector{x}, t)$ are therefore continuous but not differentiable in time for every fixed realization of $\mvector{\omega} \in \mvector{\Omega}$.

Because for every fixed realization of $\mvector{\omega} \in \mvector{\Omega}$, the RODEs \refequationp{\ref{eq-rodes}} turn into a deterministic ODE system \refequationp{\ref{eq-rodes-odes}}, one approach to solving the RODEs is to use sampling methods to first obtain many sample paths, and then solving each sample path deterministically.
In order to cover the statistics of the solution, a massive number of ODEs must be solved efficiently.
A high performance related study was conducted by \cite{riesinger2016solving}, where the sample paths are solved in parallel on a modern GPU cluster.

\section{Doss-Sussmann/Imkeller-Schmalfuss correspondence}
\label{sec-doss-sussmann}

Since a system of RODEs can be analyzed pathwise using deterministic calculus, it offers an opportunity to study its related SDEs as discussed below.

First of all, it has been shown by \cite{jentzen2011taylor} that any RODE system with a Wiener process can be expressed as its equivalent SDE system. 
Using the scalar RODE in \refequationp{\ref{eq-rodes-example}} as an example, its SDE formulation is described as
\begin{align}
    d\begin{pmatrix}
        x_t 
        \\ 
        y_t
    \end{pmatrix}
    & = 
    \begin{pmatrix}
        -x_t + \cos{(y_t)}
        \\
        0
    \end{pmatrix}
    + 
    \begin{pmatrix}
        0
        \\
        1
    \end{pmatrix}
    dW_t
\end{align}

Similarly, any finite dimensional SDE system can be transformed into its equivalent ODEs by utilizing the \emph{Doss-Sussmann/Imkeller-Schalfuss correspondence} \citep{sussmann1978gap, imkeller2001conjugacy}.
Specifically, for SDE models with additive noise, the statement from \cite[\refchapter{2}]{jentzen2011taylor} is stated as the following proposition.
\begin{proposition}
    Any finite dimensional SDE system can be transformed into an equivalent RODE system and vice versa as follows:
    \begin{align}
        \sdedx = \sdef\sdedt + \sdedwt 
        \iff 
        \frac{\rodedz}{dt} = \rodef + \rodeo
        \label{eq-rode-sde}
    \end{align}
    where $\rodez = \sdex - \rodeo$ and $\rodeo$ is a stationary stochastic \emph{Ornstein-Uhlenbeck} process defined as
    \begin{align}
        d\rodeo = -\rodeo\sdedt + \sdedwt
    \end{align}
\end{proposition}  

\section{Laplace mean-field for random dynamical systems}
\label{sec-rodes-laplace}

As discussed previously, although RODE sample paths can be analyzed using deterministic calculus, the existence of the stochastic process causes traditional numerical schemes such as the \emph{Euler} and the \emph{Runge-Kutta} methods \citep{butcher2016numerical} to fail to achieve their usual order of convergence when applied to RODEs \citep{grune2001pathwise}.
In the past, improved numerical solutions such as the integral versions of the implicit Taylor-like expansions \citep{kloeden2007pathwise} have been proposed to achieve better result.

On the other hand, since the solution paths of the RODEs are once differentiable, the gradient matching model can be ideally applied to the RODEs.
Moreover, the computational efficiency of the \algolpmf\ method allows a large number of RODE sample paths to be solved simultaneously. 
Lastly, by using the Doss-Sussmann/Imkeller-Schalfuss correspondence described before, we derive the following ensemble gradient matching algorithm, denoted as \algolpmfsde, to infer the states and the parameters of the SDEs without requiring any stochastic calculus.

\begin{algorithm}
    \centering
    \caption{Pseudocode for the \algolpmfsde\ algorithm.}
    \label{algo-lpmf-sde}
    \begin{algorithmic}[1]
        \State{Transform the SDEs into RODEs using \refequationp{\ref{eq-rode-sde}}.}
        \State{Generate $N_{paths}$ RODEs sample paths using each time an independently generated Ornstein-Uhlenbeck process sample path.}
        \For{$i = \mrange{1}{N_{paths}}$}
            \State{Estimate the states and parameters using \refalgorithm{\ref{algo-lpmf}}.}
        \EndFor
        \State{Average the estimation results from all the sample paths.}
    \end{algorithmic}
\end{algorithm}

Since the experiments are conducted with the Lorenz 63 and the Lorenz 96 models, the estimation of the parameters are carried out by extending the closed-forms solutions from \refequationp{\ref{eq-vgmgp-theta-conditional-mean}} and \refequationp{\ref{eq-vgmgp-theta-conditional-covariance}} as follows:
\begin{align}
    \dymetatheta 
    & = \dymOmegatheta \sum_k{\dymBthetakX{k}^T\dymLambdak{k}(\dymmk{k} - \dymbthetakX{k} - \mdata{O}_k)}    
    \\
    \dymXitheta^{-1}
    & = \sum_k{\dymBthetakX{k}^T\dymLambdak{k}\dymBthetakX{k}}
\end{align}
where 
\begin{align}
    \dymmk{k} &= \dymdCdphik{k}\dyminvCphik{k}\mvector{z}_k
    \\
    \dyminvLambdak{k} &= \dymdCdphik{k} - \dymdCphik{k}\dyminvCphik{k}\dymCdphik{k} + \dymgamma_k \mvector{I}
\end{align}
Similar to the notations introduced in \refsection{\ref{sec-odes}}, $\mvector{z}_u \in \R^N$ and $\mdata{Z} \in \R^{K \times N}$ refer to the states of the RODE sample path, while $\mdata{O}$ refer to the states of the Ornstein-Uhlenbeck process.
Note that the rewriting of the vector field as a linear combination of the parameters $\dymtheta$ and an extra term should satisfy
\begin{align}
    \mvector{B}_{\dymtheta k}(\mdata{Z} + \mvector{O})\dymtheta + \mvector{b}_{\dymtheta k}(\mdata{Z} + \mdata{O}) + \mdata{O} 
    = 
    \mvector{f}_k(\mdata{Z} + \mdata{O}, \dymtheta) + \mdata{O}
\end{align}
for $k = \mrange{1}{K}$.

Due to the complexity of expressing as a linear combination of the states and an extra term, the states of the RODE sample path are estimated using gradient-descent by minimizing the adapted cost function \refequationp{\ref{eq-laplace-xu-cost}} as follows:
\begin{align}
    cost_{\mvector{z}_u} 
    &= \ln{[\mathcal{N}(\mvector{z}_u\vert\mvector{\mu}(\mvector{y}_u), \dymSigmak{u})\prod_k{
        \mathcal{N}(\mvector{f}_k(\mdata{Z} + \mdata{O}, \dymtheta) + \mdata{O}\vert\dymmk{k},\dyminvLambdak{k}))]}
    }
\end{align}
for $u = \mrange{1}{K}$,  where
\begin{align}
    \dymmuk{u} &= \dymCphik{u}(\dymCphik{u} + \dymsigmak{u}^2\mI)^{-1}\dymyk{u}
    \\
    \dymSigmak{u} &= \dymsigmak{u}^2\dymCphik{u}(\dymCphik{u} + \dymsigmak{u}^2\mI)^{-1}
\end{align}
