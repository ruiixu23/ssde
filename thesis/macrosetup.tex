%% Custom commands
%% ===============

%% Special characters for number sets, e.g. real or complex numbers.
\newcommand{\C}{\mathbb{C}}
\newcommand{\K}{\mathbb{K}}
\newcommand{\N}{\mathbb{N}}
\newcommand{\Q}{\mathbb{Q}}
\newcommand{\R}{\mathbb{R}}
\newcommand{\Z}{\mathbb{Z}}
\newcommand{\X}{\mathbb{X}}

%% Fixed/scaling delimiter examples (see mathtools documentation)
\DeclarePairedDelimiter\abs{\lvert}{\rvert}
\DeclarePairedDelimiter\norm{\lVert}{\rVert}

%% Use the alternative epsilon per default and define the old one as \oldepsilon
\let\oldepsilon\epsilon
\renewcommand{\epsilon}{\ensuremath\varepsilon}

%% Also set the alternate phi as default.
\let\oldphi\phi
\renewcommand{\phi}{\ensuremath{\varphi}}

\newcommand{\mvector}[1]{\boldsymbol{#1}}
\newcommand{\mdata}[1]{\mathrm{#1}}
\newcommand{\mrange}[2]{#1,\dots,#2}
\newcommand{\mI}[0]{\mvector{I}}

\newcommand{\refchapter}[1]{Chapter #1}
\newcommand{\refchapterp}[1]{(Chapter #1)}
\newcommand{\refsection}[1]{Section #1}
\newcommand{\refsectionp}[1]{(\refsection{#1})}
\newcommand{\refequationp}[1]{(Eq.\ #1)}
\newcommand{\reffigure}[1]{Figure #1}
\newcommand{\refalgorithm}[1]{Algorithm #1}
\newcommand{\reftable}[1]{Table #1}

\DeclareMathOperator*{\argmin}{arg\,min}
\DeclareMathOperator*{\argmax}{arg\,max}
\newcommand{\defeq}{\vcentcolon=}
\newcommand{\eqdef}{=\vcentcolon}

% Y, X, dX, E
\newcommand{\dymY}[0]{\mdata{Y}}
\newcommand{\dymX}[0]{\mdata{X}}
\newcommand{\dymdX}[0]{\mdata{\dot{X}}}
\newcommand{\dymE}[0]{\mdata{E}}

\newcommand{\dymXwithoutk}[1]{\dymX_{/\{#1\}}}

\newcommand{\dymXtilde}[0]{\mdata{\widetilde{X}}}

% y(t), x(t), dx(t)
\newcommand{\dymy}[0]{\mvector{y}(t)}
\newcommand{\dymdx}[0]{\mvector{\dot{x}}(t)}
\newcommand{\dymx}[0]{\mvector{x}(t)}

% y_k, x_k, dx_k
\newcommand{\dymyk}[1]{\mvector{y}_{#1}}
\newcommand{\dymxk}[1]{\mvector{x}_{#1}}
\newcommand{\dymdxk}[1]{\mvector{\dot{x}}_{#1}}

\newcommand{\dymxtildek}[1]{\mvector{\widetilde{x}}_{#1}}

% y(t_n), x(t_n), dx(t_n)
\newcommand{\dymytn}[1]{\mvector{y}(t_{#1})}
\newcommand{\dymxtn}[1]{\mvector{x}(t_{#1})}
\newcommand{\dymdxtn}[1]{\mvector{\dot{x}}(t_{#1})}

% y_k(t_n), x_k(t_n), dx_k(t_n)
\newcommand{\dymyktn}[2]{y_{#1}(t_{#2})}
\newcommand{\dymxktn}[2]{x_{#1}(t_{#2})}
\newcommand{\dymdxktn}[2]{\dot{x}_{#1}(t_{#2})}

\newcommand{\dymxhatktn}[2]{\hat{x}_{#1}(t_{#2})}
\newcommand{\dymxtildexktn}[2]{\widetilde{x}_{#1}(t_{#2})}

% Theta
\newcommand{\dymtheta}[0]{\mvector{\theta}}
\newcommand{\dymthetam}[1]{\theta_{#1}}
\newcommand{\dymthetatilde}[0]{\mvector{\widetilde{\theta}}}
\newcommand{\dymthetatildem}[1]{\widetilde{\theta}_{#1}}

% f
\newcommand{\dymf}[0]{\mvector{f}(\dymx, \dymtheta)}
\newcommand{\dymfshort}[0]{\mvector{f}}
\newcommand{\dymfk}[1]{f_{#1}(\dymx, \dymtheta)}
\newcommand{\dymftn}[1]{\mvector{f}(\mvector{x}(t_{#1}), \dymtheta)}

\newcommand{\dymfX}[0]{\mvector{f}(\dymX, \dymtheta)}
\newcommand{\dymfkX}[1]{\mvector{f}_{#1}(\dymX, \dymtheta)}

\newcommand{\dymfkXshort}[1]{\mvector{f}_{#1}}

% t
\newcommand{\dymt}[0]{t}
\newcommand{\dymtn}[1]{t_{#1}}

% Epsilon
\newcommand{\dymepsilon}[0]{\mvector{\epsilon}}
\newcommand{\dymepsilont}[0]{\mvector{\epsilon}(t)}
\newcommand{\dymepsilontn}[1]{\mvector{\epsilon}(t_{#1})}

% Sigma
\newcommand{\dymsigma}[0]{\mvector{\sigma}}
\newcommand{\dymsigmak}[1]{\sigma_{#1}}

% Phi
\newcommand{\dymphi}[0]{\mvector{\phi}}
\newcommand{\dymphik}[1]{\mvector{\phi}_{#1}}

% Kernel
\newcommand{\dymkernel}[1]{\mathcal{K}_{\dymphik{#1}}}

% C_phi
\newcommand{\dymCphik}[1]{\mvector{C}_{\mvector{\phi}_{#1}}}
\newcommand{\dyminvCphik}[1]{\dymCphik{#1}^{-1}}
\newcommand{\dymCphikij}[1]{C_{\mvector{\phi}_{#1}i, j}}

% dC_phi
\newcommand{\dymdCphik}[1]{{}^{\prime}\mvector{C}_{\mvector{\phi}_{#1}}}
\newcommand{\dymdCphikij}[1]{{}^{\prime}C_{\mvector{\phi}_{#1}i, j}}

% Cd_phi
\newcommand{\dymCdphik}[1]{\mvector{C}^{\prime}_{\mvector{\phi}_{#1}}}
\newcommand{\dymCdphikij}[1]{C^{\prime}_{\mvector{\phi}_{#1}i, j}}

% dCd_phi
\newcommand{\dymdCdphik}[1]{\mvector{C}^{\prime\prime}_{\mvector{\phi}_{#1}}}
\newcommand{\dymdCdphikij}[1]{C^{\prime\prime}_{\mvector{\phi}_{#1}i, j}}

% mu
\newcommand{\dymmu}[0]{\mvector{\mu}}
\newcommand{\dymmuk}[1]{\dymmu_{#1}(\dymyk{#1})}

% Sigma
\newcommand{\dymSigma}[0]{\mvector{\Sigma}}
\newcommand{\dymSigmak}[1]{\dymSigma_{#1}}
\newcommand{\dyminvSigmak}[1]{\dymSigmak{#1}^{-1}}

% m
\newcommand{\dymm}[0]{\mvector{m}}
\newcommand{\dymmk}[1]{\dymm_{#1}}

% A
\newcommand{\dymAk}[1]{\mvector{A}_{#1}}

% Lambda
\newcommand{\dymLambda}[0]{\mvector{\Lambda}}
\newcommand{\dymLambdak}[1]{\dymLambda_{#1}}
\newcommand{\dyminvLambdak}[1]{\dymLambdak{#1}^{-1}}

% gamma
\newcommand{\dymgamma}[0]{\mvector{\gamma}}
\newcommand{\dymgammak}[1]{\gamma_{#1}}

% r theta
\newcommand{\dymrtheta}[0]{\mvector{r}_{\dymtheta}}

% Omega theta
\newcommand{\dymOmegatheta}[0]{\mvector{\Omega}_{\dymtheta}}
\newcommand{\dyminvOmegatheta}[0]{\dymOmegatheta^{-1}}

% r u
\newcommand{\dymru}[0]{\mvector{r}_{u}}

% Omega u
\newcommand{\dymOmegau}[0]{\mvector{\Omega}_{u}}
\newcommand{\dyminvOmegau}[0]{\dymOmegau^{-1}}

% B theta, b theta
\newcommand{\dymBthetakX}[1]{\mvector{B}_{\dymtheta{#1}}}
\newcommand{\dymbthetakX}[1]{\mvector{b}_{\dymtheta{#1}}}

% B u, b u
\newcommand{\dymBukX}[1]{\mvector{B}_{u{#1}}}
\newcommand{\dymbukX}[1]{\mvector{b}_{u{#1}}}

% Canonical
\newcommand{\dymetacanonical}[0]{\mvector{\eta}_{(\cdot)}(\cdot)}
\newcommand{\dymTcanonical}[0]{\mvector{T}_{(\cdot)}(\cdot)} 
\newcommand{\dymAcanonical}[0]{A_{(\cdot)}(\cdot)}
\newcommand{\dymhcanonical}[0]{h_{(\cdot)}(\cdot)}

\newcommand{\dymetathetacanonical}[0]{\mvector{\eta}_{\dymtheta}(\dymY,\dymX,\dymphi,\dymgamma)}
\newcommand{\dymTthetacanonical}[0]{\mvector{T}_{\dymtheta}(\dymtheta)}  
\newcommand{\dymAthetacanonical}[0]{A_{\dymtheta}(\mvector{\eta}_{\dymtheta})}
\newcommand{\dymhthetacanonical}[0]{h_{\dymtheta}(\dymtheta)}

\newcommand{\dymetaucanonical}[0]{\mvector{\eta}_{u}(\dymY,\dymXwithoutk{u},\dymphi,\dymtheta,\dymsigma,\dymgamma)}
\newcommand{\dymTucanonical}[0]{\mvector{T}_{u}(\dymxk{u})} 
\newcommand{\dymAucanonical}[0]{A_{u}(\mvector{\eta}_{u})}
\newcommand{\dymhucanonical}[0]{h_{u}(\dymxk{u})}

% Variational parameters
\newcommand{\dymlambdavi}[0]{\mvector{\lambda}}
\newcommand{\dympsivi}[0]{\mvector{\psi}}
\newcommand{\dympsiuvi}[0]{\mvector{\psi}_u}

% Canonical Q
\newcommand{\dymetathetacanonicalQ}[0]{\dymlambdavi}
\newcommand{\dymTthetacanonicalQ}[0]{\mvector{T}_{q\dymtheta}(\dymtheta)}  
\newcommand{\dymAthetacanonicalQ}[0]{A_{q\dymtheta}(\dymlambdavi)}
\newcommand{\dymhthetacanonicalQ}[0]{h_{q\dymtheta}(\dymtheta)}

\newcommand{\dymetaucanonicalQ}[0]{\dympsiuvi}
\newcommand{\dymTucanonicalQ}[0]{\mvector{T}_{qu}(\dympsiuvi)} 
\newcommand{\dymAucanonicalQ}[0]{A_{qu}(\dympsiuvi)}
\newcommand{\dymhucanonicalQ}[0]{h_{qu}(\dymxk{u})}

% Laplace approximation parameters
\newcommand{\dymetaX}[0]{\mvector{\eta}_{\dymX}}
\newcommand{\dymetaXwithoutk}[1]{\mvector{\eta}_{\dymXwithoutk{#1}}}
\newcommand{\dymetaxk}[1]{\mvector{\eta}_{\dymxk{#1}}}
\newcommand{\dymXixk}[1]{\mvector{\Xi}_{\dymxk{#1}}}
\newcommand{\dyminvXixk}[1]{\mvector{\Xi}^{-1}_{\dymxk{#1}}}

\newcommand{\dymetatheta}[0]{\mvector{\eta}_{\dymtheta}}
\newcommand{\dymXitheta}[0]{\mvector{\Xi}_{\dymtheta}}
\newcommand{\dyminvXitheta}[0]{\mvector{\Xi}^{-1}_{\dymtheta}}

%% SDE
\newcommand{\sdex}[0]{\dymx}
\newcommand{\sdedx}[0]{d\dymx}
\newcommand{\sdextn}[1]{\dymxtn{#1}}


\newcommand{\sdef}[0]{\dymf}
\newcommand{\sdeftn}[1]{\dymftn{#1}}
\newcommand{\sdefshort}[0]{\mvector{f}}

\newcommand{\sdetheta}[0]{\dymtheta}
\newcommand{\sdethetam}[1]{\dymthetam{#1}}

\newcommand{\sdedt}[0]{dt}

\newcommand{\sderho}[0]{\mvector{\rho}}
\newcommand{\sderhoq}[1]{\rho_{#1}}

\newcommand{\sdeg}[0]{\mvector{g}(\dymx, \sderho)}
\newcommand{\sdegw}[1]{\mvector{g}_{#1}(\dymx, \sderho)}
\newcommand{\sdegwtn}[2]{\mvector{g}_{#1}(\dymxtn{#2}, \sderho)}
\newcommand{\sdegshort}[0]{\mvector{g}}

\newcommand{\sdewt}[0]{\mvector{W}_t}
\newcommand{\sdewmtn}[2]{W^{#1}_{#2}}


\newcommand{\sdedwt}[0]{d\mvector{W}_t}
\newcommand{\sdedwmtn}[2]{dW^{#1}_{#2}}

\newcommand{\sdeSigma}[0]{\mvector{\Sigma}}
\newcommand{\sdeSigmaik}[2]{\Sigma_{ik}}

%% SDE and RODE
\newcommand{\probspace}[0]{(\mvector{\Omega}, \mathcal{F}, \mathbb{P})}

\newcommand{\rodez}[0]{\mvector{z}(t)}
\newcommand{\rodedz}[0]{d\rodez}
\newcommand{\rodeo}[0]{\mvector{O}_t}
\newcommand{\rodef}[0]{\mvector{f}(\rodez + \rodeo, \sdetheta)}

%% Inference algorithms
\newcommand{\algogmgp}[0]{GMGP}
\newcommand{\algovgmgp}[0]{VGMGP}

\newcommand{\algolpmf}[0]{LPMF}
\newcommand{\algolpmfpos}[0]{LPMP-POS}
\newcommand{\algolpmfsde}[0]{LPMF-SDE}
\newcommand{\algolpmfsdef}[0]{LPMF-SDE-F}
\newcommand{\algolpmfsdep}[0]{LPMF-SDE-P}

\newcommand{\algovgpa}[0]{VGPA}
\newcommand{\algovgpamf}[0]{VGPA-MF}
\newcommand{\algovgpamap}[0]{VGPA-MAP}

%% Protein signalling transduction pathway
\newcommand{\proteinS}[0]{S}
\newcommand{\proteinSdt}[0]{\dot{\proteinS}}

\newcommand{\proteindS}[0]{dS}
\newcommand{\proteindSdt}[0]{\dot{\proteindS}}

\newcommand{\proteinR}[0]{R}
\newcommand{\proteinRdt}[0]{\dot{\proteinR}}

\newcommand{\proteinRS}[0]{RS}
\newcommand{\proteinRSdt}[0]{\dot{\proteinRS}}

\newcommand{\proteinRpp}[0]{Rpp}
\newcommand{\proteinRppdt}[0]{\dot{\proteinRpp}}

\newcommand{\proteinki}[1]{k_{#1}}
\newcommand{\proteinKm}[0]{K_{m}}
\newcommand{\proteinV}[0]{V}
