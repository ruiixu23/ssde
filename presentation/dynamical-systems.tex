\begin{frame}
    \begin{center}
        {\large Dynamical Systems}
    \end{center}        
\end{frame}

\begin{frame}[t]
    \frametitle{Ordinary differential equations (ODEs)}
    A $K$-dimensional real-valued ODE system is defined as
    \begin{align}
        \dymdx & = \frac{d\dymx}{dt} = \dymf
    \label{eq-odes}
    \end{align}
    where
    \begin{itemize}
        \item[] $\dymx = [\mrange{\dymxktn{1}{}}{\dymxktn{K}{}}]^\top \in \R^K$ are the states at time $t$,
        \item[] $\dymdx = [\mrange{\dymfk{1}}{\dymfk{K}}]^\top \in \R^K$ are the  state derivatives at time $t$,
        \item[] $\dymfshort:\R^K \mapsto \R^K$ is the vector fields with parameter $\dymtheta \in \R^M$.
    \end{itemize}
    
    \vspace{\baselineskip}
    Initial states and parameters determine the future states.
        
    \vspace{1\baselineskip}
    {\footnotesize
        $\dymfshort$ may have direct dependency on $t$, which is suppressed for uncluttered notations.
    }
\end{frame}

\begin{frame}[t]
    \frametitle{Stochastic differential equations (SDEs)}
    Given a probability space $\probspace$, a $K$-dimensional SDE system with state-specific, additive Gaussian noises is defined, in the \emph{It\^{o}} form, as
    \begin{align}
        \sdedx = \sdef \sdedt + \sdeSigma^{\frac{1}{2}} \sdedwt
        \label{eq-sdes}
    \end{align} 
    where
    \begin{itemize}
    	\item[] $\sdefshort: \R^K \mapsto \R^K$ is the deterministic drift function with parameter $\sdetheta \in \R^M$
        \item[] $\sdeSigma = diag(\mrange{\rho_1^2}{\rho_K^2}) \in \R^{K \times K}$ is the diagonal noise covariance matrix,
        \item[] $\sdewt \in \R^{K}$ is a standard $K$-dimensional Wiener process.
    \end{itemize}
    
    \vspace{\baselineskip}
    Each realization is most likely a different \emph{sample path}.
    
    \vspace{\baselineskip}
    A class of multiplicative noise models can be mapped to this model.
\end{frame}

\begin{frame}
    \begin{center}
        {\large Motivation \& Challenges}
    \end{center}        
\end{frame}

\begin{frame}[t]
    \frametitle{Motivation}
    Dynamical systems model various natural phenomena in chemistry, physics, biology, economics, meteorology, etc. For example
    \begin{itemize}
        \item[-] \emph{Protein signalling transduction pathway} models the dynamics among protein species using a set of non-linear differential equations.
        \item[-] Stochastic \emph{Lorenz 96} model is commonly used in weather forecast.
        \item[-] Many others \dots
    \end{itemize}

    \vspace{\baselineskip}    
    The inference algorithm should be accurate, robust and performant.
\end{frame}

\begin{frame}[t]
    \frametitle{Challenges}
    \begin{itemize}
        \item[-] Conventional methods requires explicit numerical integrations each time after parameter adaptation, which is slow and not scalable.
        \item[-] The likelihood surfaces are likely to be multimodal due to nonlinearity within the dynamical systems, which makes parameter search difficult.
        \item[-] In Bayesian statistics, the marginalization term is intractable and requires approximate inference techniques.
        \begin{itemize}
            \item[] \emph{Markov chain Monte Carlo (MCMC)} sampling schemes are accurate but computationally expensive and requires onerous convergence analysis.
        \end{itemize}         
    \end{itemize}
\end{frame}
